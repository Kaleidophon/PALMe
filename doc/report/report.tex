\documentclass[a4paper,12pt]{scrartcl} 

\usepackage{hyperref}
\usepackage[german]{babel} 
\usepackage[utf8]{inputenc}
\usepackage{setspace}
\usepackage{cite}
\usepackage{graphicx}
\usepackage{array}
\setlength{\parindent}{0em}
\usepackage{booktabs}
\usepackage{xcolor}
\usepackage{amsmath}
\usepackage{amsfonts}
\usepackage{amssymb}
\usepackage{caption}
\usepackage[paper=a4paper,left=30mm,right=30mm,top=30mm,bottom=30mm]{geometry} 
\usepackage[nottoc, numbib]{tocbibind}
\usepackage{CJKutf8}
\usepackage[T3,OT2,T1]{fontenc} 
\usepackage[noenc]{tipa}
\usepackage{listings}
\usepackage{fancybox}
\usepackage{multirow}
\usepackage{cprotect}
\usepackage{enumerate}
\usepackage{xcolor}
\usepackage{setspace}

%Abstand der Fussnoten
\deffootnote{1em}{1em}{\textsuperscript{\thefootnotemark\ }}

\newcommand{\rr}{\raggedright}
\setcounter{secnumdepth}{3}

\newcolumntype{P}[1]{>{\raggedright\arraybackslash}p{#1}}
\makeatletter
\newcolumntype{F}[1]{>{\raggedright\arraybackslash\@minipagetrue\flushemize}p{#1}<{\endflushemize}}
\makeatother

\newenvironment{flushemize}{%
    % You should put this outside the `list`. The new environment will make them local anyway:
    \setlength{\itemsep}{0pt}%
    \setlength{\parskip}{0.1pt}%
    \setlength{\parsep}{0.1pt}%
    \setlength{\partopsep}{0.1pt}%
    \setlength{\topsep}{0pt}%
    \setlength{\leftmargin}{6pt}%
    % Use \list ... \endlist instead of \begin{list} ... \end{list} inside another environment
    \list{$\bullet$}\unskip}
    {\endlist}

\begin{document}


%Beginn der Titelseite
\begin{titlepage}
\begin{small}
\vfill {Ruprecht-Karls-Universität Heidelberg\\ 
Institut für Computerlinguistik\\ 
Sommersemester 2015\\
Advanced Programming (P III) \\
Shigehiko Schamoni}
\end{small}


\begin{center}
\begin{Large}
\vfill {\textsf{\textbf{
Ein parallelisiertes Sprachmodell
}}}
\end{Large}
\end{center}

\begin{small}
\vfill <Daten>
\today
\end{small}

\end{titlepage}
%Ende der Titelseite


%Inhaltsverzeichnis (aktualisiert sich erst nach dem zweiten Setzen)
\tableofcontents
\thispagestyle{empty}

\section{Einführung}

\section{Zielsetzung}

\section{Ressourcen}

    \subsection{Wikipedia als Korpus}

    \subsection{Hadoop}

\section{Frequenzzählung}

    \subsection{Vorgehen}

    \subsection{Experimente}

    \subsection{Zwischenfazit}

\section{Berechnung von \emph{n}-gram Wahrscheinlichkeiten}

    \subsection{Vorgehen}

    \subsection{Laden \& Speichern}

    \subsection{Parallelisierung}

\section{Evaluation des Sprachmodells} 

    \subsection{Vorgehen}

    \subsection{Parallelisierung}

\section{Fazit}

    \subsection{Bewertung des Ergebnisses}

    \subsection{Bewertung des Prozesses}



%Beginn einer neuen Seite
\clearpage

%Anderthalbzeiliger Zeilenabstand ab hier
\onehalfspacing

\pagestyle{plain}


%Beginn einer neuen Seite
\clearpage
\nocite{*}
\bibliography{hausarbeit_deeplearning.bib}{}
\bibliographystyle{plain}

\end{document}